\chapter{Marco Teórico}

\section{Introducción}

Este capítulo presenta los fundamentos teóricos y tecnológicos que sustentan el desarrollo del sistema SmartMeter2ThingsBoard-Gateway. Se abordan los conceptos relacionados con medición inteligente, el protocolo DLMS/COSEM, plataformas IoT, comunicación M2M (Machine-to-Machine) y arquitecturas de sistemas distribuidos para telemetría en tiempo real.

\section{Medidores Inteligentes}

\subsection{Definición y Características}

Los medidores inteligentes o \textit{smart meters} son dispositivos de medición electrónica que registran el consumo de energía eléctrica con granularidad temporal fina y son capaces de comunicarse bidireccionalmente con sistemas centrales \cite{smartgrid2020}. A diferencia de los medidores electromecánicos tradicionales, los medidores inteligentes ofrecen:

\begin{itemize}
    \item Medición en tiempo real de múltiples variables eléctricas
    \item Almacenamiento de perfiles de carga históricos
    \item Capacidad de comunicación remota
    \item Detección de eventos y anomalías
    \item Soporte para tarifación diferenciada
\end{itemize}

\subsection{Variables Eléctricas Monitoreadas}

Los medidores inteligentes típicamente registran las siguientes magnitudes:

\begin{table}[h]
\centering
\caption{Variables eléctricas monitoreadas}
\begin{tabular}{lll}
\toprule
\textbf{Variable} & \textbf{Unidad} & \textbf{Descripción} \\
\midrule
Voltaje (V) & Voltios & Tensión por fase (L1, L2, L3) \\
Corriente (I) & Amperios & Corriente por fase (L1, L2, L3) \\
Potencia Activa (P) & Vatios (W) & Potencia instantánea consumida \\
Potencia Reactiva (Q) & VAR & Componente reactiva de potencia \\
Factor de Potencia (FP) & Adimensional & Relación P/S \\
Frecuencia (f) & Hertz (Hz) & Frecuencia de la red \\
Energía Activa (E) & Vatios-hora (Wh) & Energía acumulada \\
\bottomrule
\end{tabular}
\end{table}

\section{Protocolo DLMS/COSEM}

\subsection{Descripción General}

DLMS (Device Language Message Specification) y COSEM (Companion Specification for Energy Metering) forman el estándar internacional IEC 62056 para comunicación con dispositivos de medición de energía. Este protocolo define:

\begin{itemize}
    \item Un modelo de objetos orientado a la interfaz para representar datos y funcionalidades
    \item Servicios de comunicación independientes del medio físico
    \item Mecanismos de seguridad y autenticación
    \item Perfiles de aplicación para diferentes casos de uso
\end{itemize}

\subsection{Arquitectura en Capas}

El protocolo DLMS/COSEM sigue una arquitectura OSI de capas:

\begin{enumerate}
    \item \textbf{Capa Física:} Define las interfaces de comunicación (óptica, RS-485, TCP/IP, etc.)
    \item \textbf{Capa de Enlace de Datos (HDLC):} Gestiona el establecimiento, mantenimiento y terminación de conexiones
    \item \textbf{Capa de Aplicación (ACSE/DLMS):} Maneja asociaciones, autenticación y servicios de lectura/escritura
\end{enumerate}

\subsection{Modelo de Objetos COSEM}

COSEM define objetos que representan datos y funcionalidades del medidor mediante:

\begin{itemize}
    \item \textbf{OBIS Codes:} Identificadores estándar para objetos (ejemplo: 1.0.1.8.0.255 para energía activa total)
    \item \textbf{Clases de Interfaz:} Definen la estructura y métodos de cada tipo de objeto
    \item \textbf{Atributos:} Datos asociados a cada objeto
    \item \textbf{Métodos:} Operaciones que pueden ejecutarse sobre los objetos
\end{itemize}

\subsection{HDLC (High-Level Data Link Control)}

HDLC es el protocolo de enlace de datos utilizado en DLMS/COSEM para comunicación punto a punto. Sus características incluyen:

\begin{itemize}
    \item Transmisión orientada a tramas con detección de errores CRC-16
    \item Control de flujo mediante ventanas deslizantes
    \item Direccionamiento de dispositivos mediante direcciones lógicas
    \item Soporte para modos balanceado y no balanceado
\end{itemize}

\subsubsection{Estructura de Trama HDLC}

\begin{verbatim}
Flag | Address | Control | HCS | Information | FCS | Flag
0x7E |  1-4B   |   1B    | 2B  |  Variable   | 2B  | 0x7E
\end{verbatim}

Donde:
\begin{itemize}
    \item \textbf{Flag:} Delimitador de inicio/fin (0x7E)
    \item \textbf{Address:} Dirección del dispositivo
    \item \textbf{Control:} Tipo y secuencia de trama
    \item \textbf{HCS:} Header Check Sequence (CRC-16 del encabezado)
    \item \textbf{Information:} Datos de la capa DLMS
    \item \textbf{FCS:} Frame Check Sequence (CRC-16 de toda la trama)
\end{itemize}

\subsection{Servicios DLMS}

Los principales servicios DLMS son:

\begin{table}[h]
\centering
\caption{Servicios DLMS principales}
\begin{tabular}{ll}
\toprule
\textbf{Servicio} & \textbf{Descripción} \\
\midrule
GET & Lectura de atributos de objetos \\
SET & Escritura de atributos de objetos \\
ACTION & Ejecución de métodos de objetos \\
GET-WITH-LIST & Lectura de múltiples atributos \\
EVENT-NOTIFICATION & Notificación asíncrona de eventos \\
\bottomrule
\end{tabular}
\end{table}

\section{Internet de las Cosas (IoT)}

\subsection{Definición y Características}

El Internet de las Cosas se refiere a la interconexión de dispositivos físicos mediante internet, permitiéndoles recopilar e intercambiar datos. Las características clave son:

\begin{itemize}
    \item Conectividad ubicua de dispositivos
    \item Capacidad de procesamiento distribuido
    \item Generación masiva de datos (Big Data)
    \item Comunicación M2M (Machine-to-Machine)
    \item Integración con servicios en la nube
\end{itemize}

\subsection{Arquitectura IoT}

Una arquitectura IoT típica comprende cuatro capas:

\begin{enumerate}
    \item \textbf{Capa de Dispositivos:} Sensores, actuadores y gateways
    \item \textbf{Capa de Red:} Protocolos de comunicación y conectividad
    \item \textbf{Capa de Plataforma:} Gestión de dispositivos, procesamiento y almacenamiento
    \item \textbf{Capa de Aplicación:} Interfaces de usuario y lógica de negocio
\end{enumerate}

\subsection{Plataformas IoT}

Las plataformas IoT proporcionan infraestructura para:

\begin{itemize}
    \item Registro y gestión de dispositivos
    \item Ingesta y procesamiento de telemetría
    \item Almacenamiento de datos time-series
    \item Motor de reglas y alertas
    \item APIs para integración
    \item Dashboards de visualización
\end{itemize}

\section{ThingsBoard}

\subsection{Descripción General}

ThingsBoard es una plataforma IoT de código abierto diseñada para recopilación, procesamiento, visualización y gestión de dispositivos IoT. Está construida con tecnologías escalables y ofrece:

\begin{itemize}
    \item Arquitectura de microservicios
    \item Soporte multi-tenancy
    \item Procesamiento de eventos en tiempo real
    \item Motor de reglas visual
    \item APIs REST y WebSocket
\end{itemize}

\subsection{Arquitectura de ThingsBoard}

ThingsBoard utiliza una arquitectura basada en:

\begin{itemize}
    \item \textbf{PostgreSQL:} Base de datos relacional para metadatos
    \item \textbf{TimescaleDB:} Extensión de PostgreSQL para series temporales
    \item \textbf{Apache Kafka:} Sistema de mensajería para procesamiento asíncrono
    \item \textbf{Redis:} Caché en memoria para sesiones y datos temporales
    \item \textbf{Rule Engine:} Motor de reglas para procesamiento de eventos
\end{itemize}

\subsection{Modelo de Datos}

ThingsBoard organiza los datos mediante:

\begin{itemize}
    \item \textbf{Dispositivos:} Entidades que envían telemetría
    \item \textbf{Assets:} Representaciones lógicas de entidades físicas
    \item \textbf{Relaciones:} Vínculos entre dispositivos y assets
    \item \textbf{Telemetría:} Datos de series temporales
    \item \textbf{Atributos:} Metadatos estáticos o semi-estáticos
\end{itemize}

\section{Protocolo MQTT}

\subsection{Descripción General}

MQTT (Message Queuing Telemetry Transport) es un protocolo de mensajería ligero diseñado para comunicación M2M en entornos con ancho de banda limitado. Sus características principales son:

\begin{itemize}
    \item Arquitectura Publish-Subscribe
    \item Bajo overhead de protocolo
    \item Niveles de Quality of Service (QoS)
    \item Soporte para Last Will and Testament (LWT)
    \item Persistencia de sesiones
\end{itemize}

\subsection{Niveles de QoS}

MQTT define tres niveles de calidad de servicio:

\begin{table}[h]
\centering
\caption{Niveles de QoS en MQTT}
\begin{tabular}{cll}
\toprule
\textbf{QoS} & \textbf{Nombre} & \textbf{Garantía} \\
\midrule
0 & At most once & Sin confirmación, puede perderse \\
1 & At least once & Con confirmación, puede duplicarse \\
2 & Exactly once & Protocolo de 4 vías, entrega única \\
\bottomrule
\end{tabular}
\end{table}

En este proyecto se utiliza \textbf{QoS 1} para garantizar la entrega de telemetría sin el overhead del QoS 2.

\subsection{Tópicos MQTT}

Los tópicos en MQTT siguen una estructura jerárquica separada por barras:

\begin{verbatim}
v1/devices/me/telemetry
v1/devices/me/attributes
v1/gateway/telemetry
\end{verbatim}

\section{Arquitecturas de Telemetría}

\subsection{Patrón Gateway}

El patrón Gateway es fundamental en sistemas IoT donde dispositivos sin conectividad directa a internet requieren un intermediario. Características:

\begin{itemize}
    \item Agregación de datos de múltiples dispositivos
    \item Transformación de protocolos
    \item Buffering local para tolerancia a desconexiones
    \item Procesamiento edge (en el borde)
\end{itemize}

\subsection{Time-Series Databases}

Las bases de datos de series temporales están optimizadas para:

\begin{itemize}
    \item Ingestión masiva de datos ordenados temporalmente
    \item Consultas con agregaciones temporales (promedio, máximo, mínimo)
    \item Retención de datos con políticas de compactación
    \item Particionamiento temporal automático
\end{itemize}

\subsection{Arquitectura Lambda}

La arquitectura Lambda combina procesamiento batch y streaming:

\begin{itemize}
    \item \textbf{Batch Layer:} Procesamiento de grandes volúmenes históricos
    \item \textbf{Speed Layer:} Procesamiento en tiempo real
    \item \textbf{Serving Layer:} Unificación de resultados batch y streaming
\end{itemize}

\section{Resiliencia y Recuperación}

\subsection{Patrones de Recuperación}

Los sistemas de telemetría críticos requieren mecanismos de recuperación:

\begin{itemize}
    \item \textbf{Retry con backoff exponencial:} Reintentos con espera creciente
    \item \textbf{Circuit Breaker:} Prevención de llamadas a servicios fallidos
    \item \textbf{Bulkhead:} Aislamiento de recursos para evitar fallos en cascada
    \item \textbf{Timeout:} Límites de tiempo para operaciones
\end{itemize}

\subsection{Health Checks}

Los health checks permiten monitorear el estado del sistema:

\begin{itemize}
    \item Verificación de conectividad con dispositivos
    \item Monitoreo de latencias y timeouts
    \item Detección de degradación de rendimiento
    \item Métricas de tasa de éxito/fallo
\end{itemize}

\section{Contenedorización con Docker}

\subsection{Descripción General}

Docker es una plataforma de contenedorización que permite empaquetar aplicaciones con sus dependencias en unidades portables. Beneficios:

\begin{itemize}
    \item Aislamiento de procesos
    \item Portabilidad entre entornos
    \item Escalabilidad horizontal
    \item Gestión declarativa de infraestructura
\end{itemize}

\subsection{Docker Compose}

Docker Compose permite definir aplicaciones multi-contenedor mediante archivos YAML:

\begin{verbatim}
services:
  thingsboard:
    image: thingsboard/tb-postgres:4.2.1
    environment:
      - TB_QUEUE_TYPE=kafka
    depends_on:
      - postgres
      - kafka
\end{verbatim}

\section{Seguridad en IoT}

\subsection{SSL/TLS}

El protocolo TLS (Transport Layer Security) proporciona:

\begin{itemize}
    \item Encriptación de comunicaciones
    \item Autenticación mutua mediante certificados
    \item Integridad de mensajes
    \item Prevención de ataques man-in-the-middle
\end{itemize}

\subsection{Autenticación y Autorización}

Los sistemas IoT implementan múltiples niveles de seguridad:

\begin{itemize}
    \item Autenticación basada en tokens
    \item Control de acceso basado en roles (RBAC)
    \item API keys con scopes limitados
    \item Rotación automática de credenciales
\end{itemize}

\section{Estado del Arte}

\subsection{Soluciones Comerciales}

Existen soluciones comerciales para integración DLMS/COSEM:

\begin{itemize}
    \item \textbf{Gurux DLMS/COSEM:} Librería comercial .NET
    \item \textbf{Trilliant SecureMesh:} Sistema propietario AMI
    \item \textbf{Landis+Gyr Gridstream:} Plataforma end-to-end
\end{itemize}

Limitaciones: Alto costo, falta de flexibilidad, dependencia de proveedor.

\subsection{Proyectos de Código Abierto}

Proyectos open-source relacionados:

\begin{itemize}
    \item \textbf{python-dlms-cosem:} Librería Python parcialmente implementada
    \item \textbf{Node-RED DLMS:} Nodos básicos para Node-RED
    \item \textbf{OpenMUC DLMS:} Framework Java para sistemas SCADA
\end{itemize}

Limitaciones: Implementaciones incompletas, falta de integración con plataformas IoT modernas.

\subsection{Contribución de este Trabajo}

Este proyecto aporta:

\begin{itemize}
    \item Implementación completa y documentada del stack DLMS/COSEM
    \item Integración nativa con plataforma IoT moderna (ThingsBoard)
    \item Arquitectura resiliente con recuperación automática
    \item Sistema completo de administración y monitoreo
    \item Código abierto y extensible
\end{itemize}

\section{Conclusiones del Capítulo}

Este capítulo ha presentado los fundamentos teóricos que sustentan el desarrollo del sistema SmartMeter2ThingsBoard-Gateway. Se han cubierto aspectos relacionados con:

\begin{itemize}
    \item El protocolo DLMS/COSEM y su implementación técnica
    \item Arquitecturas IoT y plataformas de gestión
    \item Protocolos de comunicación M2M como MQTT
    \item Patrones de diseño para sistemas de telemetría resilientes
    \item Estado actual de soluciones comerciales y open-source
\end{itemize}

Estos conceptos forman la base sobre la cual se diseñó e implementó la arquitectura del sistema, que se describe en detalle en el siguiente capítulo.
