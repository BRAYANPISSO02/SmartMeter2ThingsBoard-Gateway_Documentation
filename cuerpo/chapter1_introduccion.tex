\chapter{Introducción}

\section{Contexto y Motivación}

La transformación digital de la infraestructura eléctrica representa uno de los desafíos más importantes en la actualidad. Los medidores inteligentes o \textit{smart meters} han revolucionado la manera en que se mide, monitorea y gestiona el consumo de energía eléctrica. Sin embargo, la integración de estos dispositivos en sistemas IoT (Internet de las Cosas) capaces de procesar, almacenar y visualizar datos en tiempo real sigue siendo un reto técnico significativo.

El protocolo DLMS/COSEM (Device Language Message Specification / Companion Specification for Energy Metering) es el estándar internacional IEC 62056 utilizado para la comunicación con medidores inteligentes. A pesar de ser ampliamente adoptado en la industria, su implementación presenta desafíos relacionados con la complejidad del protocolo, la gestión concurrente de múltiples dispositivos y la necesidad de garantizar alta disponibilidad y confiabilidad en la adquisición de datos.

Por otro lado, ThingsBoard es una plataforma IoT de código abierto que ofrece capacidades avanzadas de gestión de dispositivos, procesamiento de telemetría, visualización de datos y generación de alertas. Sin embargo, ThingsBoard no cuenta con soporte nativo para el protocolo DLMS/COSEM, lo que crea una brecha tecnológica que limita su aplicación en el sector de medición inteligente.

Este proyecto aborda esta problemática mediante el desarrollo de un sistema integral que actúa como puente entre medidores inteligentes DLMS/COSEM y la plataforma ThingsBoard IoT, habilitando así capacidades de monitoreo remoto, análisis histórico y gestión centralizada de infraestructuras de medición eléctrica.

\section{Planteamiento del Problema}

La implementación de sistemas de telemetría para medidores inteligentes enfrenta múltiples desafíos:

\begin{enumerate}
    \item \textbf{Complejidad del protocolo DLMS/COSEM:} El estándar IEC 62056 define una estructura de comunicación compleja que requiere manejo especializado de tramas HDLC (High-Level Data Link Control), gestión de asociaciones ACSE (Association Control Service Element) y decodificación de objetos COSEM.
    
    \item \textbf{Gestión concurrente de múltiples dispositivos:} Los sistemas de medición eléctrica típicamente involucran decenas o cientos de medidores que deben ser monitoreados simultáneamente, lo que requiere arquitecturas escalables y resilientes.
    
    \item \textbf{Confiabilidad y recuperación ante fallos:} Las redes de comunicación con medidores pueden presentar interrupciones, desconexiones temporales o errores de transmisión que deben ser manejados de manera robusta para garantizar la integridad de los datos.
    
    \item \textbf{Integración con plataformas IoT:} La ausencia de conectores nativos para DLMS/COSEM en plataformas IoT populares dificulta la implementación de soluciones de monitoreo y análisis avanzado.
    
    \item \textbf{Visualización y análisis en tiempo real:} Los operadores de sistemas eléctricos requieren dashboards intuitivos, alertas automáticas y capacidades de análisis histórico para la toma de decisiones operativas.
\end{enumerate}

\section{Objetivos}

\subsection{Objetivo General}

Desarrollar e implementar un sistema integral de telemetría IoT que permita la comunicación bidireccional entre medidores inteligentes basados en el protocolo DLMS/COSEM y la plataforma ThingsBoard, habilitando el monitoreo remoto en tiempo real, el almacenamiento histórico y la visualización de variables eléctricas críticas.

\subsection{Objetivos Específicos}

\begin{enumerate}
    \item Diseñar e implementar un orquestador de telemetría capaz de gestionar múltiples medidores DLMS/COSEM de manera concurrente, garantizando alta disponibilidad y recuperación automática ante fallos.
    
    \item Desarrollar un sistema de adquisición de datos que implemente el estándar IEC 62056 completo, incluyendo autenticación, encriptación y gestión de sesiones HDLC/ACSE.
    
    \item Implementar mecanismos de Quality of Service (QoS) nivel 1 para garantizar la entrega confiable de mensajes de telemetría mediante el protocolo MQTT.
    
    \item Diseñar y desplegar una infraestructura ThingsBoard contenerizada con soporte para múltiples protocolos de comunicación (MQTT, HTTP, CoAP) y capacidades de procesamiento de eventos en tiempo real.
    
    \item Crear dashboards de visualización personalizables que permitan el monitoreo de variables eléctricas (voltaje, corriente, potencia, energía) con alertas automáticas ante condiciones anormales.
    
    \item Implementar herramientas de administración y monitoreo que incluyan APIs REST, interfaces CLI y sistemas de logging estructurado para facilitar la operación y el mantenimiento del sistema.
    
    \item Validar el sistema mediante pruebas de integración, carga y resiliencia que demuestren su capacidad para operar en entornos de producción.
\end{enumerate}

\section{Alcance del Proyecto}

El proyecto \textbf{SmartMeter2ThingsBoard-Gateway} comprende el desarrollo completo de dos componentes principales:

\subsection{Componente 1: DLMS Telemetry Orchestrator}

Sistema de software desarrollado en Python que incluye:

\begin{itemize}
    \item Implementación completa del protocolo DLMS/COSEM según estándar IEC 62056
    \item Orquestador multi-medidor con soporte para operación concurrente
    \item Sistema de recuperación automática con tres niveles de resiliencia
    \item Cliente MQTT con QoS nivel 1 para transmisión confiable de telemetría
    \item Base de datos SQLite para gestión de configuración y estado
    \item API REST para control remoto y administración
    \item Dashboard web para monitoreo y configuración
    \item Sistema de alertas y notificaciones automáticas
    \item Herramientas CLI para gestión operativa
\end{itemize}

\subsection{Componente 2: ThingsBoard Telemetry Platform}

Infraestructura IoT contenerizada que incluye:

\begin{itemize}
    \item Plataforma ThingsBoard 4.2.1 con arquitectura de microservicios
    \item Base de datos PostgreSQL 16 para almacenamiento time-series
    \item Apache Kafka para procesamiento asíncrono de eventos
    \item Redis para caché y gestión de sesiones
    \item Soporte SSL/TLS para comunicaciones seguras
    \item Conectores para múltiples protocolos (MQTT, HTTP, CoAP/LwM2M)
    \item Dashboards personalizables de visualización
    \item Sistema de reglas y alertas configurable
\end{itemize}

\subsection{Limitaciones}

\begin{itemize}
    \item El sistema está diseñado y probado específicamente con medidores Microstar compatibles con DLMS/COSEM
    \item La comunicación se realiza mediante interfaz óptica (puerto óptico) y no incluye comunicación remota (GPRS/LTE)
    \item El proyecto se enfoca en variables eléctricas básicas (voltaje, corriente, potencia, energía) sin incluir mediciones avanzadas de calidad de energía
    \item El alcance no incluye integración con sistemas SCADA o ERP empresariales
\end{itemize}

\section{Justificación}

El desarrollo de este sistema se justifica por múltiples razones:

\subsection{Necesidad Tecnológica}

La carencia de soluciones integradas que conecten medidores DLMS/COSEM con plataformas IoT modernas representa una barrera para la modernización de infraestructuras de medición eléctrica. Este proyecto llena ese vacío tecnológico proporcionando una solución completa y documentada.

\subsection{Aplicabilidad Industrial}

Los medidores inteligentes son ampliamente utilizados en:
\begin{itemize}
    \item Empresas de distribución eléctrica
    \item Grandes consumidores industriales
    \item Edificios inteligentes y campus universitarios
    \item Microrredes y sistemas de generación distribuida
    \item Proyectos de investigación en eficiencia energética
\end{itemize}

\subsection{Aporte Académico}

Este trabajo contribuye al conocimiento en áreas de:
\begin{itemize}
    \item Implementación de protocolos industriales complejos
    \item Diseño de sistemas IoT escalables y resilientes
    \item Arquitecturas de software para telemetría en tiempo real
    \item Integración de tecnologías de código abierto para aplicaciones industriales
\end{itemize}

\subsection{Beneficios Esperados}

\begin{itemize}
    \item Reducción de costos operativos mediante monitoreo remoto
    \item Mejora en la toma de decisiones mediante análisis de datos históricos
    \item Detección temprana de anomalías mediante alertas automáticas
    \item Optimización del consumo energético mediante visualización en tiempo real
    \item Escalabilidad para gestionar desde unidades individuales hasta grandes parques de medidores
\end{itemize}

\section{Metodología}

El desarrollo del proyecto se realizó siguiendo una metodología iterativa e incremental basada en las siguientes fases:

\subsection{Fase 1: Análisis y Diseño}
\begin{itemize}
    \item Estudio del estándar IEC 62056 y protocolo DLMS/COSEM
    \item Análisis de especificaciones técnicas de medidores Microstar
    \item Diseño de arquitectura de software del sistema completo
    \item Definición de casos de uso y requisitos funcionales
\end{itemize}

\subsection{Fase 2: Implementación del Orquestador DLMS}
\begin{itemize}
    \item Desarrollo del cliente DLMS/COSEM base
    \item Implementación del sistema multi-medidor
    \item Integración con cliente MQTT y mecanismos de QoS
    \item Desarrollo de sistema de recuperación ante fallos
\end{itemize}

\subsection{Fase 3: Desarrollo de Herramientas de Administración}
\begin{itemize}
    \item Implementación de API REST
    \item Desarrollo de dashboard web
    \item Creación de herramientas CLI
    \item Sistema de logging y monitoreo
\end{itemize}

\subsection{Fase 4: Despliegue de Infraestructura ThingsBoard}
\begin{itemize}
    \item Configuración de arquitectura contenerizada con Docker
    \item Implementación de certificados SSL/TLS
    \item Configuración de conectores y reglas de procesamiento
    \item Desarrollo de dashboards de visualización
\end{itemize}

\subsection{Fase 5: Integración y Pruebas}
\begin{itemize}
    \item Pruebas de integración entre componentes
    \item Pruebas de carga y rendimiento
    \item Pruebas de resiliencia y recuperación
    \item Validación en entorno de producción
\end{itemize}

\subsection{Fase 6: Documentación}
\begin{itemize}
    \item Documentación técnica del sistema
    \item Guías de instalación y operación
    \item Documentación de APIs
    \item Elaboración de este documento de trabajo de grado
\end{itemize}

\section{Organización del Documento}

Este documento está estructurado de la siguiente manera:

\begin{itemize}
    \item \textbf{Capítulo 2 - Marco Teórico:} Presenta los fundamentos teóricos relacionados con el protocolo DLMS/COSEM, plataformas IoT, comunicación MQTT y arquitecturas de telemetría.
    
    \item \textbf{Capítulo 3 - Arquitectura del Sistema:} Describe en detalle la arquitectura de software y hardware del sistema, incluyendo diagramas, componentes y patrones de diseño utilizados.
    
    \item \textbf{Capítulo 4 - Implementación:} Explica las decisiones técnicas, algoritmos implementados y detalles de desarrollo de cada componente del sistema.
    
    \item \textbf{Capítulo 5 - Pruebas y Resultados:} Presenta las pruebas realizadas, métricas obtenidas y análisis de rendimiento del sistema.
    
    \item \textbf{Capítulo 6 - Conclusiones y Trabajo Futuro:} Resume los logros del proyecto, presenta conclusiones y propone líneas de trabajo futuro.
    
    \item \textbf{Anexos:} Incluye documentación técnica adicional, diagramas detallados, ejemplos de código y guías de usuario.
\end{itemize}
