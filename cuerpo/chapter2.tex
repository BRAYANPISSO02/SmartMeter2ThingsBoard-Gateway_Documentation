% called by main.tex
%
\chapter{Estado del arte}
\label{ch::chapter2}
\section{Cómo usar esta plantilla}

Esta plantilla ha sido desarrollada siguiendo las guías de formato de las tesis doctorales de la UPM, que - en muchos aspectos - siguen estándar de formato comunes a muchas instituciones. 

El documento se divide en tres componentes (preliminares, cuerpo y  anexos), cada uno de ellos con diferentes subcomonentes, como se muestra en la Tabla \ref{tab:thesisComp}:

\begin{table}[h]
\centering
    \caption{Componentes del documento del Trabajo de Fin de Master}
    \label{tab:thesisComp}
\begin{tabularx}{\textwidth}{l l}
\toprule
\textbf{Componentes} & \textbf{Subcomponentes}\\
\midrule
\multirow{9}{10em}{Preliminares} & Cubierta \\ 
 & Portada \\
  & Página de créditos \\
  & [Dedicatoria, Agradecimientos] \\
    & Abstract/Resumen \\
     & Índice general \\
     & Índice de figuras  \\
     & Índice de tablas \\
     & Abreviaturas y acrónimos \\
\midrule 
\multirow{3}{10em}{Cuerpo} & Introducción \\ 
 & Capítulos centrales específicos para cada trabajo\\
 & Conclusiones \\
\midrule
\multirow{2}{10em}{Anexos} & Referencias \\ 
     & Anexos \\
\bottomrule
    \end{tabularx}
\end{table}

La mayoría de las normas de formato se refieren a los preliminares; esto es, a las páginas iniciales que aparecen antes de los capítulos de la tesis (el cuerpo principal de la tesis). 

A continuación se presentan las normas de obligado seguimiento y algunas recomendaciones.

\section{Normas de obligado seguimiento}
Las \textbf{reglas principales} a tener en cuenta son:
\begin{itemize}
    \item Las tres primeras páginas (cubierta, portada y página de créditos) deben seguir el formato especificado en esta plantilla. 
    \item Las páginas iniciales son obligatorias, exceptuando la dedicatoria y los agradecimientos. 
    \item El trabajo debe ser presentada en formato electrónico (PDF).
    \item Idioma: el trabajo se puede escribir en español o en inglés.
    \item Tamaño de página: A4
    \item Numeración: Los preliminares - el bloque antes del inicio de los capítulos - usa numeración romana, con los números centrados en la parte inferior, excluyendo la cubierta y la portada, que no deben estar numeradas. El resto del texto (comenzando con la Introducción) usa numeración arábiga, con los números de página centrados en la parte inferior.

\end{itemize}

La \textbf{cubierta}:
\begin{itemize}
    \item Debe prepararse utilizando la plantilla presente, introduciendo los datos apropiados en las secciones marcadas con <>, evitando incluir información adicional. 
    \item Debe incluir el nombre oficial del centro, en español y sin abreviaturas (véase \href{https://www.upm.es/sfs/Rectorado/Vicerrectorado%20de%20Investigacion/Doctorado/Tesis/impresos/5_Listado%20de%20PD%20y%20sus%20centros.pdf}{Lista}).
    \item Debe incluir el nombre completo del estudiante de máster. 
\end{itemize}

La \textbf{portada}:
\begin{itemize}
    \item  Debe ser preparada utilizando la plantilla presente, introduciendo la información apropiada en las secciones marcadas con < > evitando incluir información adicional.
    \item Debe incluir el nombre oficial del centro, en español y sin abreviaturas (véase \href{https://www.upm.es/sfs/Rectorado/Vicerrectorado%20de%20Investigacion/Doctorado/Tesis/impresos/5_Listado%20de%20PD%20y%20sus%20centros.pdf}{Lista}). Puedes incluir el logotipo del centro, respectando el tamaño indicado en la plantilla (véase \href{https://www.upm.es/UPM/SalaPrensa/IdentidadGrafica/LogosPlantillas}{Logotipos}).
    \item Debe incluir el nombre completo del programa de máster (Máster en Aprendizaje Automático y Datos Masivos)).
    \item Debe incluir el nombre completo del estudiante de máster.
    \item Debe incluir el nombre completo del director de trabajo (y de co-director en caso de haberlo).
\end{itemize}

La \textbf{página de créditos} (página i):
\begin{itemize}
    \item Debe prepararse usando esta plantilla, introduciendo los datos apropiados en las secciones marcadas con < > evitando incluir información adicional. 
    \item Incluye información adicional en el campo del director del trabajo: la posición y la institución (ídem en el campo del co-director en caso de haberlo).
    \item Las secciones “Revisores Externos”, “Tribunal de Trabajo de Fin de Master” y “Fecha de Defensa” se dejan vacías.
    \item Si el trabajo ha recibido financiación por alguna convocatoria competitiva, debe ser indicado al final de esta página (véase la plantilla).
\end{itemize}

El \textbf{resumen}:
\begin{itemize}
    \item Debe incluirse un resumen tanto en español como en inglés, independientemente del idioma principal del trabajo. 
    \item Formato: máximo 4000 caracteres, texto plano (sin símbolos).
    \item Estructura: el resumen es una presentación de la tesis y; en consecuencia, debe tener una estructura clara, incluyendo introducción (o motivación), objetivos, desarrollo y conclusiones.
\end{itemize}

\section{Recomendaciones}
En general, el cuerpo principal de la tesis cubre diversos capítulos, cuyo número y estructura variará dependiendo del campo de conocimiento. A modo de guía, esta plantilla incluye los capítuls siguientes: Introducción, Estado del arte, Material y métodos, Resultados, Discusión y Conclusiones. La estructura del cuerpo principal de cada tesis en específico debe consultarse con los supervisores del trabajo. 

Idealmente, cada capítulo se organizará en secciones y subsecciones, con cabeceras numeradas.

Como tipo de letra, se recomienda utilizar los estilos Times New Roman, Century, Arial, Book Antiqua, o uno similar. 

\begin{figure}[h]
\centering
    \includegraphics[width=0.7\textwidth]{figures/latexthesis.png}
\caption{Escribiendo tu trabajo}
\label{fig:phd1}
\end{figure}

Usar una plantilla ayuda a usar el mismo formato a lo largo del trabajo completo. Se recomienda usar siempre el mismo estilo al incluir citas en el texto, como también recomiendan otros autores 
(~\cite{bellDoingYourResearch2010, carterIgnoringMePart2017, odenaHowDoctoralStudents2017};~\cite{riveracaminoComoEscribirPublicar2014}). Para referenciar incluyendo el nombre del autor, puedes hacerlo así ~\citet{ahmadHowWriteDoctoral1969}. Para referenciar tablas (véase la Tabla\ref{tab:thesisComp}) y para las figuras (véase la Figura\ref{fig:phd1}). Se recomienda numerar las tablas y las figuras en función de los capítulos (Tabla 1.1, etc.).

La parte final del trabajo incluye las referencias y los anexos.

Las referencias deben ser incluidas usando el estilo recomendado para cada campo de conocimiento.

Los anexos incluyen material adicional no incluido en el cuerpo principal (cuestionarios, resultados adicionales, etc.). Se recomienda numerar los anexos de manera alfabética (A, B, ...) y comenzar cada anexo en una página diferente. En caso de incluir tablas en los anexos, se debe usar una numeración independiente de la usada en el cuerpo principal de la tesis (Tabla A.1., etc.).



